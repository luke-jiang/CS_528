\documentclass{article}
\usepackage[utf8]{inputenc}

\title{CS528 Assignment II}
\author{Luke Jiang, jiang700@purdue.edu }
\date{February 2022}

\begin{document}

\maketitle

\section{Question 1}
\begin{itemize}
    \item for key length of $80$: $\frac{2^{80}~(keys)}{2^{60}~(keys/sec)} = 2^{20}~sec = \frac{2^{20}}{24*60*60}~days \approx 12~days$
    \item for key length of $128$: $\frac{2^{128}~(keys)}{2^{60}~(keys/sec)} = 2^{68}~sec = \frac{2^{68}}{24*60*60*365}~years \approx 10*10^{12}~years$
\end{itemize}
\newpage

\section{Question 2}
\begin{itemize}
    \item The missed security principle is the principle of least privilege. Accessing SSH version should be a privileged operation and unverified clients should not have such privilege.
    \item The missed security principle is the principle of complete mediation. The servers' privileges are cached and not authenticated upon each access.
\end{itemize}
\newpage

\section{Question 3}
\begin{itemize}
    \item a) Second Preimage Resistant: Since the solution contains only one answer to be encrypted, Alice only needs to make sure that it's infeasible for Bob to come up with a different solution whose hashed value collides with Alice's hashed solution.
    \item b) Preimage Resistant: Since the hash value is read-only, an attacker must make sure that the modified binary files produce the same hash value. Otherwise, the modifications will be detected by the system.
\end{itemize}
\newpage

\section{Question 4}
\begin{itemize}
    \item a) An attacker may send a package containing a very long \verb|prev_end| field, which causes content of the subsequent packages (whose \verb|fp->offset| is smaller than \verb|prev_end|) discarded by the program.
    \item b) We can replace the old data with the new data after alignment.
\end{itemize}
\newpage

\section{Question 5}
\begin{itemize}
    \item a) Yes, it can. The adversary can fill the third block, which is within the comment field, with desired header plaintext and use the produced cyphertext directly as the header of the attacking package.
    \item b) No. 
\end{itemize}
\newpage

\section{Question 6}
\begin{itemize}
    \item a) Yes. since the recipient, the message and the nonce are all signed by A, B can ensure that A is sending the message to it with message m.
    \item b) No. An attacker can intercept A's reply, modify the message and send the tampered message to B.
    \item c) No. An attacker can intercept A's reply, replace the original encoded message with a tampered message encoded using B's public key and send it to B.
    \item d) No. An attacker can intercept A's reply, decode the message and nonce with the B's public key, replace the message and encode with B's public key again before sending it to B.
    \item e) Yes. Even though the recipient is not included in the message, B can still verify that the message is meant to be sent to it because the nonce is signed by A.
\end{itemize}


\end{document}
